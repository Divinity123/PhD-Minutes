\documentclass{article}
\usepackage[margin=1in]{geometry}
\title{Minutes from 27 September}

\usepackage{float}
\usepackage{amsmath}

\begin{document}
\section{Papers referred to}
\begin{enumerate}
	\item All previous papers on attacks on databases
	\item Marie-Sarah Lacharit´e, Kenneth G. Paterson. \textbf{Frequency-smoothing encryption: preventing snapshot attacks on encrypted databases}
\end{enumerate}


\section{Formalising the frequency smoothing problem}
\begin{itemize}
	\item We consider the static case, i.e. we are given the realization of the database (not a distribution) and try to hide frequencies of the attributes,
	
	\item The main techniques identified are cutting (reducing height of the high-frequent attributes) and padding.
	
	\item Correlation between columns need to be considered too.
\end{itemize}


\section{A new OPE scheme}
\begin{itemize}
	\item Modified the construction so that the claim made in the earlier meeting is indeed met. \textbf{Modification: } instead of only merging disjoint blocks at each level, the `sliding window' only moves by one block at each time. The time/space complexity stays the same because the modification only results in an expansion factor of space by 2.
	
	\item Security of this scheme is not analysed yet. (\textbf{To think: } is security with a simulator proof (even perfectly defined) sufficient to ensure security in practice?)
\end{itemize}


\section{A weaker OPE scheme}
\begin{itemize}
	\item \textbf{Idea: } instead of having all the blocks to answer queries, we use only the bottom layer.
	
	\item In that case, relative order of the blocks will be revealed with queries. But how do we quantify leakage in this case?
\end{itemize}

\end{document}