\documentclass{article}
\usepackage[margin=1in]{geometry}
\title{Minutes from 4 October}

\usepackage{float}
\usepackage{amsmath}

\begin{document}
\section{Papers referred to}
\begin{enumerate}
	\item Marie-Sarah Lacharit´e, Kenneth G. Paterson. \textbf{Frequency-smoothing encryption: preventing snapshot attacks on encrypted databases}
	
	\item Thomas Baign`eres, Pascal Junod, and Serge Vaudenay. \textbf{How Far Can We Go Beyond Linear Cryptanalysis?}
\end{enumerate}


\section{FSE revisited}
\begin{itemize}
	\item The security model measures the advantage of the adversary differentiating the resulting distribution from uniform distribution but the KL divergence used is an approximation to some other value ($P_e$) in the second paper listed above.
	
	\item The encryption scheme does not consider correlation between columns, i.e. encryption is done column by column.
	
	\item Maybe correlation can be exploited to attack the scheme. (\textbf{See folder `Attack on FSE'})
\end{itemize}



\section{More on formalising the frequency smoothing problem}
\begin{itemize}
	\item Cutting and padding are operations which will cause expansion of data size. We need to define a cost function to measure the performance of our frequency smoothing algorithm.
	
	\item We need to define security more carefully: we may not need uniformly distributed histogram by the end, e.g. a histogram that has two levels and attributes are randomly assigned to each one of them.
\end{itemize}


\end{document}