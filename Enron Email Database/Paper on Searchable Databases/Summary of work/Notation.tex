The following notations are used for sets we used to describe a searchable database.
\begin{table}[H]
	\centering
	\begin{tabular}{ll}
		\textbf{Notation} & \textbf{Explanation} \\ \hline
		$\DB$     	  	  & Set of searchable databases \\
		$\EDB$    		  & Set of encrypted searchable databases \\
		$\KWset$		  & Set of keywords in a database \\
		$\EKWset$		  & Set of encrypted keywords in a encrypted searchable databases \\
		$\EIndex$		  & Set of encrypted indices for keyword searching \\
		$\File$			  & Set of files \\
		$\EFile$		  & Set of encrypted files \\
		$\Query$		  & Set of queries on the database \\
		$\EQuery$		  & Set of encrypted queries on encrypted database \\
		$\QState$		  & Set of states to remember the query \\
		$\Response$		  & Set of responses from a query \\
		$\EResponse$	  & Set of encrypted responses from an encrypted query
	\end{tabular}
\end{table}


The following notations are used to describe realisations of a database.
\begin{table}[H]
	\centering
	\begin{tabular}{ll}
		\textbf{Notation} & \textbf{Explanation} \\ \hline
		$\db$     	  	  & Database that has not been encrypted \\
		$\edb$    		  & Encrypted database \\
		$\kwset$		  & Keyword set associated to some file \\
		$\kw$			  & A keyword from keyword set $\KWset$ \\
		$\ekwset$		  & Encrypted set associated to some encrypted file \\
		$\ekw$			  & An encrypted keyword from encrypted keyword set $\EKWset$ \\
		$\eindex$		  & An encrypted index associated to an encrypted database \\
		$\file$			  & A file in the database \\
		$\efile$		  & An encrypted file \\
		$\query$		  & A query on the database \\
		$\equery$		  & An encrypted query on the encrypted database \\
		$\qState$		  & A state to remember the query \\
		$\response$		  & A response from a query \\
		$\eResponse$	  & An encrypted response from an encrypted query
	\end{tabular}
\end{table}

We use $\Pi_i$ to mean projection of a product of sets to its $i$-th component. For example, $\Pi_1(A \times B) = A$. We abuse $\kwset(\file)$ and $\File(\kwset)$ to mean set of keywords associated to file $\file$ and the set of files associated to keyword set $\kwset$. Similarly, we abuse $\ekwset(\efile)$ and $\EFile(\kwset)$ to mean set of encrypted keywords associated to encrypted file $\efile$ and the set of encrypted files associated to keyword set $\kwset$. We may write $\KWset(\db)$ to mean the set of keywords associated to the database $\db$ and $\EKWset(\edb)$ to mean the set of encrypted keywords associated to the encrypted database $\edb$.