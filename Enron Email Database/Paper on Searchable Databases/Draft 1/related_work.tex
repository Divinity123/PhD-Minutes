
In this section, we give a brief account of constructions of searchable encryption scheme, known attacks on them, and counter-measures proposed to negate the attacks. We argue that the current counter-measures are insufficient to protect the underlying data.

\subsection{Keyword Search in Encrypted Databases}
There is a wide range of schemes proposed to allow for keyword search in encrypted databases \cite{SP:FVYSHG17}. 

Schemes which use symmetric key primitives are known as SSE schemes. Song et al. \cite{SP:SonWagPer00} proposed the first SSE scheme. The construction essentially puts keywords into a chain and encrypt it that way, such that the whole chain has to be visited when a search query is issued. The scheme is proven to be a secure encryption scheme but it is not proven to be a secure encryption scheme. In fact, frequency of the underlying plaintexts are revealed upon searching so the scheme is vulnerable to statistical attacks such as inference analysis. More recent works such as \cite{EPRINT:Goh03, EPRINT:ChaMit04, CCS:CGKO06, FC:KamPap13, NDSS:StePapShi14, NDSS:CJJJKR14, SP:NavPraGun14, CCS:HahKer14, IWSEC:KSOY16, RSA:Gajek16, CCS:KKLPK17} proposes to build encrypted inverted index for the keywords in the document collection. The time complexity of search query is proportional to the number of keywords in the database. In terms of security notion, Goh \cite{EPRINT:Goh03} proposed semantic security against adaptive chosen keyword attack (IND-CKA and IND2-CKA), and Chang and Mitzenmacher \cite{EPRINT:ChaMit04} introduced a simulation-based definition that is slightly stronger. Both of the definitions are shown to be insufficient in \cite{CCS:CGKO06}. The authors in \cite{CCS:CGKO06} has also formulated new security notions both in non-adaptive setting and in adaptive setting.

On the other hand, schemes which use public key primitives are known as PEKS schemes. The first PEKS scheme was proposed by Boneh et al. \cite{EC:BDOP04}. Later on, Bellare et al. \cite{C:BelBolONe07} introduced the notion of public key efficiently searchable encryption (ESE) and proposed constructions in the random oracle model. The main difference between public key ESE and symmetric key solutions is that public key ESE uses deterministic encryption on keywords so the frequency of keywords are leaked before any queries. Subsequent work focus on enhancing the security notion of ESE and to protect the database from keyword guessing attack.

Recently, there has been an extensive work attempting to enrich expressiveness of the search function. Fuzzy keyword search enhances functionality by allowing non-exact matches. For example, search of `rael' returns documents containing `real' too. Fuzzy keyword search has been studied in symmetric key setting in \cite{EPRINT:LWWCRL09, 5462196, EPRINT:XuJin10, ACISP:ZMWLC17} and in public key setting in \cite{5199004, EPRINT:XuJin10}. Conjunctive keyword search allows the user to search for multiple keywords. It has been studied in symmetric key settings in \cite{ACNS:GolStaWat04, ICICS:BalKamMon05, CANS:WanWanPie08, C:CJJKRS13, ESORICS:FJKNRS15} and in public key setting in \cite{WISA:ParKimLee04, TCC:BonWat07, SP:SBCSP07, PAIRING:HwaLee07, EC:LOSTW10, JC:KatSahWat13, ASIACCS:LZDLC13}. It is possible for the server to be dishonest and return only partial or inaccurate results. Verifiable keyword search is devised to tackle the problem. In symmetric key setting \cite{1320027, Li2008, 6364125, 7035110} are proposed. In public key setting \cite{EPRINT:ZheXuAte13, 7016139} are proposed.



\subsection{Attacks on Current Schemes}
There are many known genres of attacks on searchable encryption. In this paper, we focus on the attacks abusing keyword frequencies in the database. Some of the schemes leak frequencies before any queries, while some other schemes only leak frequencies of the queried keywords. We stress that the leakage profile in the latter case is not good enough to protect the database against statistical attack, since with enough queries, the adversary can recover frequency of all keywords.

The first frequency-based attack on searchable encryption, known as IKK attack, is presented by Islam et al. \cite{NDSS:IslKuzKan12}. In their attack model, the adversary is assumed to know the distribution of number of documents of pairs of keywords and his goal is to recover trapdoors used in queries. By comparing the distribution to the observed counts in queries using simulated annealing, the authors were able to recover almost all queries for relatively small ($\sim$500) keyword set size, and for keyword set size of 2500, their attack could successfully identify most of the queries. The efficiency and accuracy of the attack is later improved by Pouliot et al. \cite{CCS:PouWri16} by casting the problem as a graph matching problem.

By observing that a large fraction of keywords match to unique number of documents, Cash et al. proposed count attack \cite{CCS:CGPR15}. Their attack achieves better recovery rate of queries compared to IKK attack. Naveed et al. \cite{CCS:NavKamWri15} have demonstrated that even simple frequency analysis is sufficient to recover encrypted keywords in databases.

\textbf{TODO:} one more attack proposed in 2017

\subsection{Countermeasures}
Islam et al. \cite{NDSS:IslKuzKan12} suggested $(\alpha, t)-$secure index to prevent inference attack. The scheme aims to have partitions of keywords of size at least $\alpha$ each, such that in each partition, the sets of documents containing those keywords differ by at most $t$ documents. They proposed an algorithm to achieve $(\alpha, 0)$-secure index for any databases. The scheme uses clustering algorithm on the database so it is static in nature.

Lacharit{\'e} et al. proposed to use frequency-smoothing encryption to defend against frequency analysis \cite{cryptoeprint:2017:1068}. The authors have only considered single-column encryption in relational databases so the scheme is not provably secure against attacks based on co-occurrence of keywords. Bost et al. \cite{EPRINT:BosFou17} attempted to fix security notion for searchable encryption by posting additional constraints. They proposed a padding-based scheme and tested the scheme against count attack \cite{CCS:CGPR15}. The experiment results show that their scheme is still vulnerable to count attack.

To our best knowledge, there is yet a satisfactory security notion for searchable encryption that is secure against statistical attacks (leakage-abuse attacks in general), and there is no scheme known to be secure under all statistical attacks. 









