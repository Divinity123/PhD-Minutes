In this section, we will define permutable matrix. In short, permutable matrix is a special type of matrix with the property that after applying row and column permutations, we get the same matrix as before. After that, we generalize the idea to higher dimensions.

\begin{definiton}
	Let $M$ be a two-dimensional $n$ by $n$ matrix. Let $P,Q$ be permutation matrices on $[n]$. Then we say $M$ is $(P,Q)$-permutable if $PMQ^{T} = M$. If $P=Q$, we simply say $M$ is $P$-permutable.
\end{definiton}

We stress that the following is not a proof of security. Imagine we have $G$ an co-occurrence count matrix on keywords $(m_1, \cdots m_k)$. The adversary knows $G$ and a permutation of it, say $H = QGQ^{T}$ for some permutation matrix $Q$. His goal is to recover $Q$ so he can map the encrypted keywords to the plaintexts. Indeed, we know from \cite{CCS:PouWri16} that this can be done with high accuracy. We imagine the worst case where the adversary recovers $Q$ completely. If $Q$ is the unique solution to $\min_{X} \norm{XGX^{T} - H}$, then the adversary can recover all keywords with certainty. However, if $G$ is $P$-permutable, we see $\norm{QGQ^{T} - H} = \norm{QPGP^{T}Q^{T} - H} = \norm{(QP)G(QP)^{T} - H}$, so $QP$ also minimizes the objective function. In fact, all permutations of the shape $QP^i$ for $i = 0, \cdots, k-1$ minimizes the norm. In that case, the adversary can no longer map keywords to their encryptions uniquely. Since $P$ is a permutation of full length, $QP^i$ sends keywords to different encryptions for each $i$. Hence, if we start with an co-occurrence count matrix that permutable by some permutation of full length, we can significantly reduce the likelihood of the adversary recovering encryptions of keywords via co-occurrence based attacks.

\textbf{Probably have to change after settle on security notion.} The argument above is not a proof of security for three reasons. First of all, co-occurrence matrix itself is aggregated information, we cannot assume that it is the only information the adversary uses. Secondly, there is no reason to assume that co-occurrence matrix based attack is the only technique that can be used to breach security. Finally, the goal of the adversary is not necessarily keyword recovery. For instance, he can ask the following question: `Is keyword $\kw$ in either file $\file_0$ or file $\file_1$?'.







